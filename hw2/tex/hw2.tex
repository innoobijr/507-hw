\documentclass{handout}

\usepackage{listings}
\usepackage{mathpartir}
\usepackage{hyperref}
\usepackage{color}
\usepackage{amsmath}
\usepackage{graphicx}
\usepackage[T1]{fontenc}


\definecolor{blue}{RGB}{0,128,200}
\definecolor{green}{RGB}{0,128,0}
\definecolor{gray}{RGB}{64, 64, 64}

\hypersetup{
colorlinks=true,
filecolor=blue,
urlcolor=blue,
citecolor=blue,
linkcolor=blue}

\SetCourseTitle{CSE 507:  Computer-Aided Reasoning for Software}
\SetInstructor{Zachary Tatlock}
\SetSemester{Fall 2023}

\newcommand\HW[1]{\SetHandoutTitle{Homework Assignment #1}}
\newcommand\Due[1]{\SetDueDate{#1 at 5:00pm}}

\newcommand\gitlab{https://gitlab.cs.washington.edu/cse507/hw23au}
\newcommand\gitlabpath{\gitlab/tree/main}
\newcommand\website{https://courses.cs.washington.edu/courses/cse507/23au}
\newcommand\rosette{\href{https://emina.github.io/rosette/}{Rosette}}
\newcommand\cadical{\href{https://github.com/arminbiere/cadical}{CaDiCaL}}
\newcommand\dimacs{\href{https://www.satcompetition.org/2011/format-benchmarks2011.html}{DIMACS}}


\lstloadlanguages{Lisp}
\lstdefinestyle{inline}{numbers=none,basicstyle=\ttfamily}
\newcommand\cc[1]{{\lstset{style=inline}\lstinline{#1}}}
\newcommand\ccm[1]{{\lstset{style=inline}\lstinline[mathescape]|#1|}}


\HW{2}
\Due{November 10, 2023}

\begin{document}
\maketitle

\begin{tabular}{ll}
\textbf{Total points:} & 100 \\
\textbf{Deliverables:}
& \texttt{hw2.pdf} containing typeset solutions to Problems \ref{prob:smt:first}-\ref{prob:alloy:last}.\\
& \texttt{verifier.rkt} containing your implementation for Problem \ref{prob:bvv:last}.\\
& \texttt{uf.rkt} containing your implementation for Problems \ref{prob:uf:first}-\ref{prob:uf:last}.\\
& \texttt{nnf.als} containing your Alloy encoding for Problems \ref{prob:alloy:first}-\ref{prob:alloy:last}.\\
\textbf{Sources:}
& \href{\gitlab}{\gitlab}.\\
\end{tabular}


\section{SMT Solving (30 points)}

\begin{questions}
\item (5 points) \label{prob:smt:first} Apply the congruence closure algorithm
to decide the satisfiability of the following $T_=$ formula:
\[
f(g(x)) = g(f(x)) \wedge f(g(f(y))) = x \wedge f(y) = x \wedge g(f(x)) \neq x
\]
Provide the level of detail as in \lecture{07}.  In particular, show the
intermediate partitions (sets of congruence classes) after each merge or
propagation step, together with a brief explanation of how the algorithm arrived
at that partition (e.g., ``by literal $f(x) = y$, merge $f(x)$ with $y$'').
\label{prob:first}




\item Consider the following formula in $T_=\cup T_R$:
\[
g(x + y, z) = f(g(x, y)) \wedge x + z = y \wedge z \geq 0 \wedge x \geq y \wedge g(x, x) = z \wedge f(z) \neq g(2x, 0)
\]

\begin{enumerate}
\item (5 points) Purify the formula and show the resulting $T_=$ and $T_R$
formulas. Show the purification results using the table below. Apply
purification to the (current) innermost term first.  If there are several
innermost terms, prefer the leftmost one.  Use $a_i$ to refer to the
$i^\text{th}$ auxiliary literal, starting with $a_1$.  All occurrences of the
same term should be mapped to the same auxiliary literal. You do not need to
show the individual steps of the purification process, just the final result.

\[
\begin{array}{@{}l|l}
T_=           & T_R\\ \hline
\ldots & \ldots \\
\end{array}
\]



\item (5 points) Use the Nelson-Oppen procedure to decide the satisfiability of
the purified formula.  In one sentence, state which version of the procedure you
are using and why.  Show the equality propagation  by filling out the table
below.  If $T_i$ infers the $j^\text{th}$ equality (or disjunction of
equalities), enter it into the $j^\text{th}$ row and $i^\text{th}$ column
only---leave the remaining column in that row empty.

\[
\begin{array}{@{}l|l}
T_= & T_R\\ \hline
\ldots & \ldots \\
\end{array}
\]


\end{enumerate}

\item (5 points) Recall that the theory of arrays $T_A =
\{\emph{read},\emph{write},=\}$ is defined by the following axioms.

\[
\begin{array}{@{}l}
\forall a,i,j. \ i = j \rightarrow \emph{read}(a, i) = \emph{read}(a, j)\\
\forall a,v,i,j. \ i = j \rightarrow \emph{read}(\emph{write}(a, i, v), j) = v\\
\forall a,v,i,j. \ i \neq j \rightarrow \emph{read}(\emph{write}(a, i, v), j) = \emph{read}(a, j)\\
\end{array}
\]

Prove that $T_A$ is not convex by constructing $n\geq 3$ formulas in $T_A$ such
that $F_1 \Rightarrow (F_2 \vee\ldots \vee F_n)$ but $F_1 \not\Rightarrow F_i$
for any $i \in [2\ldots n]$.




\item (10 points) \label{prob:smt:last} Prove that the theory of equality $T_=$
is convex.





\end{questions}



\section{A Verifier for Superoptimization (25 points)}

\emph{Superoptimization} is the task of replacing a given loop-free sequence of
instructions with an equivalent sequence that is better according to some metric
(e.g., shorter).  Modern superoptimizers work by employing various forms of the
guess-and-check strategy:  given a sequence $s$ of instructions, they guess a
better replacement sequence $r$, and then they check that $s$ and $r$ are
equivalent.  In this problem, you will develop a simple SMT-based verifier for
superoptimization.  Given two loop-free sequences of 32-bit integer
instructions, your verifier will either confirm that they are equivalent or, if
they are not, it will produce a concrete counterexample---an input on which the
two sequences produce different outputs.

The verifier will accept programs in the \texttt{BV} language, which has the
following grammar:

{\tt\small
\begin{tabular}{lcl}
Prog &:=&  (\textbf{define-fragment} (id id$^\mathtt{*}$) Stmt$^\mathtt{*}$ Ret) \\
Stmt &:=& (\textbf{define} id Expr) $|$ (\textbf{set!}\ id Expr) \\
Ret &:=& (\textbf{return} Expr) \\
Expr &:=& id $|$ const $|$ (\textbf{if} Expr Expr Expr) $|$ (unary-op Expr) $|$ \\
         &&(binary-op Expr Expr) $|$ (nary-op Expr$^\mathtt{+}$)\\
unary-op &:=& bvneg $|$ bvnot \\
binary-op &:=& = $|$ bvule $|$ bvult $|$ bvuge $|$ bvugt $|$ bvsle $|$ bvslt  $|$ bvsge $|$ bvsgt $|$  \\
                && bvsdiv $|$ bvsrem $|$ bvshl $|$ bvlshr $|$ bvashr $|$ bvsub\\
nary-op &:=& bvor $|$ bvand $|$ bvxor $|$  bvadd $|$  bvmul\\
id &:=& identifier\\
const &:=& 32-bit integer $|$ true $|$ false\\
\end{tabular}}

Assume the following well-formedness rules for programs, which your verifier
does not need to check:
\begin{enumerate}
\item an identifier is not used before it is defined;
\item an identifier is not defined more than once;
\item the first sub-expression of an if-expression is of type boolean, and its remaining subexpressions have the same type.
\end{enumerate}
The statement \texttt{(\textbf{set!}\ id Expr)}  assigns the value of
\texttt{Expr} to the variable \texttt{id}; the types of \texttt{id} and
\texttt{Expr} must match.  The inputs to a fragment are 32-bit integers.

The operators in the \texttt{BV} language have the same semantics as the
corresponding operators in $T_\emph{bv}$ (see the
\href{http://rise4fun.com/z3/tutorial}{Z3 tutorial} on bitvectors). For example,
these are valid \texttt{BV} programs:

\lstset{language=lisp}
\lstset{alsoletter={-}}
\lstset{morekeywords={return, define, define-fragment}}
\begin{lstlisting}
(define-fragment (P1 x1 y1 x2 y2)
  (return (bvmul (bvadd x1 y1) (bvadd x2 y2))))
\end{lstlisting}

\begin{lstlisting}
(define-fragment (P2 x1 y1 x2 y2)
  (define u1 (bvadd x1 y1))
  (define u2 (bvadd x2 y2))
  (return (bvmul u1 u2)))
\end{lstlisting}


\begin{questions}

\item (5 points) \label{prob:bvv:first} The grammar for the \texttt{BV} language
is designed in such a way that you do not need to convert a \texttt{BV} program
to Static Single Assignment (SSA) form before translating it to bit vector
logic.  Explain in a few sentences what property of this grammar allows you to
avoid SSA conversion.




\item (20 points) \label{prob:bvv:last} Implement a BMC-style verifier for the
\texttt{BV} language in \racket, using the solution skeleton in the \src{bvv}
directory. Your verifier (see \src[bvv/]{verifier.rkt}) should take as input two
\texttt{BV} program fragments (see \src[bvv/]{examples.rkt} and
\src[bvv/]{bv.rkt}); produce a  \texttt{QF\_BV} formula that is unsatisfiable
iff  the programs are equivalent; invoke Z3 on the generated formula
(\src[bvv/]{solver.rkt}); and decode Z3's output as follows.  If the programs
are equivalent, the verifier should return \texttt{'EQUIVALENT}; otherwise it
should return an input, expressed as a list of integers, on which the fragments
produce a different output.

Inputs to the two programs should be the only unknowns (i.e., bitvector
constants) in the \texttt{QF\_BV} formula produced by your verifier.  This means
that the verifier cannot use additional constants to represent the values of
program expressions and statements.  But it should also not inline the
translations of individual expressions.  For example, consider the following
\texttt{BV} fragment:


\begin{lstlisting}
(define-fragment (toy b c)
  (define a (bvmul b c))
  (return (bvadd a a)))
\end{lstlisting}
\label{prob:impl}

The encoding may introduce two unknowns to represent the input variables
\texttt{b} and \texttt{c}.  But it may not translate the first statement by
emitting an SMT-LIB equality assertion such as $(\mathsf{assert}\ (= a\
(\mathsf{bvmul}\ b\ c)))$, where $a$ is a fresh unknown.  Similarly, it may not
translate the return statement by inlining the encoding of the first statement,
i.e., $(\mathsf{bvadd}\ (\mathsf{bvmul}\ b\ c)\ (\mathsf{bvmul}\ b\ c))$.

\medskip
(\textbf{Hint:}  Your encoding may use SMT-LIB definitions, introduced by \texttt{define-fun}.)
\medskip

Your encoding should be entirely contained in the \src[bvv/]{verifier.rkt} file.
In particular, the \texttt{verify-all} procedure in \src[bvv/]{tests.rkt} should
be executable just by placing your \src[bvv/]{verifier.rkt} into the \src{bvv}
directory, without modifying any supporting files.  Your encoding will be tested
and graded automatically, so it is important for the implementation to be
self-contained, and to adhere to the input/output specification given above.
Note also that we will test your code on benchmarks that are not included in
\src[bvv/]{tests.rkt}.  To make sure that your verifier works correctly, you
will need to write additional tests of your own, especially for corner cases.





\end{questions}


\section{Faster Verification With(out) Uninterpreted Functions (20 points)}

One common application of uninterpreted functions, shown in \lecture{06}, is to
accelerate verification by abstracting away the details of operations that are
expensive to compute.  In this part of the assignment, you will use
uninterpreted functions to accelerate two verification queries in a \rosette\
program.  Then, you will accelerate these queries without uninterpreted
functions.

The program and the queries are given in \src[uf/]{matrix.rkt}.  The program is
a straightforward implementation of matrix multiplication, and the queries
verify that the implementation satisfies two simple properties: \texttt{query-1}
checks that $A = B \implies A*B = B*A$, and \texttt{query-2} checks that $B =
A^T \implies (A*A)^T = B*B$.

Both queries take a long time to solve even on small symbolic matrices, e.g.,
$4\times4$.  The goal is to get each query to complete in seconds on
$20\times20$ matrices.  We will take two different approaches to reach this
goal.

\begin{questions}
\item \label{prob:uf:first} (10 points) The first approach is to abstract a set
of operators of your choosing with uninterpreted functions as described in
\src[uf/]{uf.rkt}.  This set should include as few operators as possible.

Your solution must be fully contained in your copy of \src[uf/]{uf.rkt}.  No
other files may be modified.  Submit your copy of \src[uf/]{uf.rkt}, along with
the typeset (short) answers to the following questions.

What is the \emph{minimal} set of operators you need to abstract with
uninterpreted functions in order to accelerate both queries?   Do your
abstractions depend on any properties of these operators?\looseness=-1



\item \label{prob:uf:last} (10 points)  Next, suppose that you cannot use
uninterpreted functions, but you are allowed to rewrite each query to solve an
\emph{equivalent} verification problem.  Briefly describe how to rewrite
\texttt{query-1} and \texttt{query-2} into logically equivalent queries that
achieve the same level of performance as in Problem \ref{prob:uf:first}, and
briefly argue why your rewriting strategy results in equivalent verification
problems. (\textbf{Hint:}  thinking about equalities is key to solving this
problem.)



\end{questions}

\section{Finite Model Finding with Alloy (25 points)}

In this part of the assignment, you will formalize the syntax and semantics of
propositional logic in \alloy, and use this formalization to check the theorem
about the monotonicity of negation normal form (NNF) from Problem 3 in
\href{\website/doc/hw1.pdf}{Homework 1}.\looseness=-1

To start, download
\href{https://github.com/AlloyTools/org.alloytools.alloy/releases/download/v5.0.0.1/Alloy-5.0.0.1.jar}{Alloy-5.0.0.1.jar}
and double click on it to launch the Alloy Analyzer tool.  You may also want to
skim Parts
\href{http://alloytools.org/tutorials/day-course/s1_logic.pdf}{1}--\href{http://alloytools.org/tutorials/day-course/s2_language.pdf}{2}
of the Alloy \href{http://alloytools.org/tutorials/day-course/}{tutorial} and
this
\href{http://homepage.cs.uiowa.edu/~tinelli/classes/181/Fall17/Notes/alloy-cheatsheet.pdf}{quick
reference} on Alloy syntax and semantics.

A skeleton solution can be found in \src[nnf/]{nnf.als}.  Complete the missing
definitions and submit your copy of \src[nnf/]{nnf.als}.  Solutions will be
automatically checked against a reference specification, so they need to be
fully contained in the submitted file.


\begin{questions}
\item \label{prob:alloy:first} (10 points) Start by formalizing the syntax of propositional logic and NNF.
\begin{enumerate}
\item A \emph{formula} in propositional logic can be a variable, negation,
conjunction, or disjunction. A formula has zero or more children, depending on
its type (e.g., variables have no children).  We represent formulas as abstract
syntax graphs that allow sharing of subformulas (so two different formulas may
contain the same subformula) but disallow cycles (i.e., no formula contains
itself). Formalize these constraints by completing the definition of the
\texttt{wellFormedFormulas} fact. Use the Alloy Analyzer to check that your
definition is correct (e.g., it rejects variables with children) and non-vacuous
(i.e., it admits some formulas) in a universe with a small number of formulas.

\item An NNF formula satisfies additional constraints on its structure.  What
are those constraints?     Formalize them by completing the definition of the
\texttt{NNF} predicate.  Check your definition for correctness and vacuity bugs.


\end{enumerate}

\item (10 points) Next, formalize the semantics of propositional logic.
\begin{enumerate}
\item Specify the semantics by completing the definition of the
\texttt{valuation} predicate. This predicate takes as input a relation from
formulas to booleans, and returns true iff the input relation is a total
function that maps each formula to its meaning. (\textbf{Hint:} to specify a
formula's meaning, consider when the semantics function should map each kind of
formula to \texttt{True}.)

\item Use the \texttt{valuation} predicate to define the \texttt{satisfies}
predicate that specifies what it means for a valuation to satisfy a given
formula.
\end{enumerate}
Check your definitions for correctness and vacuity bugs.


\item (5 points) \label{prob:alloy:last} Finally, finish formalizing the theorem
about the monotonicity of NNF from Problem 3 in
\href{\website/doc/hw1.pdf}{Homework 1}. To do so, complete the definition of
the \texttt{pos} function so that it returns the positive  set of a valuation
with respect to a formula. Check that the theorem holds for all formulas in a
small universe. To guard against vacuity, also ensure that the Alloy Analyzer
can find some formula and valuations that satisfy the premise and the
conclusion of the theorem.\looseness=-1

\end{questions}


\end{document}
