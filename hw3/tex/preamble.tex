\usepackage{listings}
\usepackage{hyperref}
\usepackage{color}
\usepackage{amsmath}
\usepackage{graphicx}
\usepackage{wrapfig}
\usepackage[T1]{fontenc}
\usepackage[scaled]{beramono}
\usepackage{xspace}
\usepackage{enumitem}

\definecolor{blue}{RGB}{0,128,200}
\definecolor{green}{RGB}{0,128,0}
\definecolor{gray}{RGB}{64, 64, 64}

\hypersetup{
colorlinks=true,
filecolor=blue,
urlcolor=blue,
citecolor=blue,
linkcolor=blue}

\SetCourseTitle{CSE 507:  Computer-Aided Reasoning for Software}
\SetInstructor{Zachary Tatlock}
\SetSemester{Fall 2023}

\newcommand\hw{}
\newcommand\HW[1]{\renewcommand\hw{#1}\SetHandoutTitle{Homework Assignment #1}}
\newcommand\Due[1]{\SetDueDate{#1 at 5:00pm}}

\newcommand\gitlab{https://gitlab.cs.washington.edu/cse507/hw23au}
\newcommand\gitlabpath{\gitlab/tree/main}
\newcommand\website{https://courses.cs.washington.edu/courses/cse507/23au}
\newcommand\rosette{\href{https://emina.github.io/rosette/}{Rosette}}
\newcommand\cadical{\href{https://github.com/arminbiere/cadical}{CaDiCaL}}
\newcommand\dimacs{\href{https://www.satcompetition.org/2011/format-benchmarks2011.html}{DIMACS}}

\newcommand\racket{\href{http://racket-lang.org}{Racket}}
\newcommand\alloy{\href{http://alloytools.org}{Alloy}}

\newcommand\dafny{\href{https://github.com/dafny-lang/dafny}{Dafny}\xspace}
\newcommand\imp{\textsc{Imp}\xspace}
\newcommand\ivl{\textsc{IVL}\xspace}

\newcommand\lecture[2][pdf]{\href{\website/doc/L#2.#1}{Lecture #2}}
\newcommand\src[2][]{\href{\gitlab/-/tree/main/hw\hw/#1#2}{\texttt{#2}}}
\newcommand\file[1]{\src[#1]{#1}}



\lstloadlanguages{Lisp}
\lstset{language=lisp}
\lstset{linewidth=3in}
\lstset{xleftmargin=0.3in}
\lstset{frame=none}
\lstset{framesep=0.1in}
\lstset{captionpos=b}
\lstset{basicstyle=\small\ttfamily}
\lstset{keywordstyle=\ttfamily\bfseries}
\lstset{commentstyle=\color{DarkGreen}}
\lstset{alsoletter={-,\#,:,=}}
\lstset{morekeywords={define,begin,if,while,\#:claim,skip,abort,:=,assume,assert,havoc,\#:invariant,\#:requires,\#:ensures,return}}
\lstset{deletekeywords={reverse}}
\lstset{backgroundcolor={}}
\lstset{mathescape=true}
\lstdefinestyle{inline}{numbers=none}
\newcommand\cc[1]{{\lstset{style=inline}\lstinline{#1}}}
\newcommand\ccm[1]{{\lstset{style=inline}\lstinline[mathescape]|#1|}}
