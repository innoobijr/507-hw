\documentclass{handout}

\usepackage{listings}
\usepackage{mathpartir}
\usepackage{hyperref}
\usepackage{color}
\usepackage{amsmath}
\usepackage{graphicx}
\usepackage[T1]{fontenc}


\definecolor{blue}{RGB}{0,128,200}
\definecolor{green}{RGB}{0,128,0}
\definecolor{gray}{RGB}{64, 64, 64}

\hypersetup{
colorlinks=true,
filecolor=blue,
urlcolor=blue,
citecolor=blue,
linkcolor=blue}

\SetCourseTitle{CSE 507:  Computer-Aided Reasoning for Software}
\SetInstructor{Zachary Tatlock}
\SetSemester{Fall 2023}

\newcommand\HW[1]{\SetHandoutTitle{Homework Assignment #1}}
\newcommand\Due[1]{\SetDueDate{#1 at 5:00pm}}

\newcommand\gitlab{https://gitlab.cs.washington.edu/cse507/hw23au}
\newcommand\gitlabpath{\gitlab/tree/main}
\newcommand\website{https://courses.cs.washington.edu/courses/cse507/23au}
\newcommand\rosette{\href{https://emina.github.io/rosette/}{Rosette}}
\newcommand\cadical{\href{https://github.com/arminbiere/cadical}{CaDiCaL}}
\newcommand\dimacs{\href{https://www.satcompetition.org/2011/format-benchmarks2011.html}{DIMACS}}


\lstloadlanguages{Lisp}
\lstdefinestyle{inline}{numbers=none,basicstyle=\ttfamily}
\newcommand\cc[1]{{\lstset{style=inline}\lstinline{#1}}}
\newcommand\ccm[1]{{\lstset{style=inline}\lstinline[mathescape]|#1|}}


\HW{3}
\Due{December 01, 2023}

\begin{document}
\maketitle

\begin{tabular}{ll}
\textbf{Total points:} & 100 \\
\textbf{Deliverables:}
& \texttt{hw3.pdf} containing typeset solutions to Problems \ref{prob:imp-first}--\ref{prob:imp-last}, \ref{prob:hoare-study}, and \ref{prob:ver-study}.\\
& \texttt{proofs.rkt} containing your proof outlines for Problem \ref{prob:hoare-proofs}.\\
& \texttt{ivl.rkt} containing your \imp to \ivl translations for Problems \ref{prob:ivl-inv}-\ref{prob:ivl-unroll}.\\
& \texttt{seqlib.dfy} containing your Dafny code and annotations for Problems~\ref{prob:dafny-first}-\ref{prob:dafny-last}.\\
\end{tabular}

\section{Implementing \imp (10 points)}\label{sec:imp}

In Sections~\ref{sec:imp:hoare}-\ref{sec:imp:wp/sp/se} of this assignment, you
will use and implement several verification tools for \imp, the simple
imperative language introduced in \lecture{11}. To help you get started, we have
provided an implementation of \imp in \rosette. In particular,
\src[imp/]{imp.rkt} contains an interpreter for \imp programs. The interpreter
assumes the following s-expression syntax for \imp, where $F^*$ stands for zero
or more repetitions of the form $F$:


{\tt\small
\begin{tabular}{lcl}
$P$ &:=&  (\textbf{procedure} $f$ ($x^*$) \textbf{:} ($x^*$) $S$) \\
$S$ &:=&  (\textbf{skip}) $|$ (\textbf{abort}) $|$  (\textbf{:=}\ $x$ $E$) $|$
          (\textbf{if} $C$ $S$ $S$) $|$ (\textbf{while} $C$ $S$) $|$
          (\textbf{begin} $S^*$)  \\
$C$ &:=&  \textbf{true} $|$ \textbf{false} $|$ (\textbf{!} $C$) $|$
          (\textbf{\&\&} $C^*$) $|$ (\textbf{\textbar\textbar} $C^*$) $|$
          (\textbf{=>} $C$ $C$) $|$ (\textbf{<=>} $C$ $C$) $|$ \\
    &&    (\textbf{=} $E$ $E$) $|$ (\textbf{<=} $E$ $E$) $|$ (\textbf{>=} $E$ $E$) $|$
          (\textbf{<} $E$ $E$) $|$ (\textbf{>} $E$ $E$) \\
$E$ &:=&  $x$ $|$ $Z$ $|$ (\textbf{+} $E^*$) $|$ (\textbf{*} $E^*$) $|$
          (\textbf{-} $E$ $E^*$)  \\
$Z$ &:=& integer literal \\
$x$ &:=& variable identifier \\
$f$ &:=& procedure identifier \\
\end{tabular}}


The file \src[imp/]{examples.rkt} shows how to run the \imp interpreter on a few
sample programs.

To implement and use tools for \imp, you will need to understand the provided
semantics in detail. To help with this, study the code in \src[imp/]{imp.rkt}
and answer the following questions.

\begin{questions}
  \item (2 points) \label{prob:imp-first} The \imp grammar uses the
  \lstinline{begin} form to compose zero or more \imp statements into a single
  statement. What is the meaning of the statements \lstinline{(begin)} and
  \lstinline{(begin $S$)} for any $S$?\\
  
  \textbf{Answer:} When a \lstinline{begin} statement is evaluated with no arguments it returns the current environment. If it is evaluated with a sequence of statements. The interpreter is recursively applied to the new arguments. It is essentially a fold-left where the procedure being called is the \setlist{interpretS}. Once the inner-most statement is evaluates it returns an environment and that environment is used to evaluate the next statement. If it fails on any \lstinline{S} then the interpreter aborts and returns.   

  \item (4 points) The \imp interpreter requires every variable to be assigned
  before it is referenced, and it also implements simple lexical scoping. Some
  \imp statements create local scopes, and variables defined (i.e., assigned
  only) in a local scope cannot be accessed outside of it. Which \imp statements
  (\lstinline{if}, \lstinline{while}, or \lstinline{begin}) introduce local
  scopes and what part of the statement has its own local scope?

  \textbf{Answer:} In \lstinline{IMP} \lstinline{begin} is the identifier used to indicate the creation of scope for \lstinline{IMP} programs. \lstinline{while} statements  introduce local scope for the evaluation of each iteration of the body (the conditions of the \lstinline{if} statement.  If the condition is true, it created a local scope where an invariant (\lstinline{B}) is evaluated. If the condition is false it return the environment. Is the list of statements does not have a return) the loop will not terminate. 

  \item (4 points) \label{prob:imp-last}  What are the possible outcomes of
  interpreting an arbitrary \imp program on arbitrary inputs? Assume that the
  program is syntactically well-formed; it obeys the \imp scoping rules; and it
  is given the right number of inputs according to its signature.\\

  \textbf{Answer:}\\
  
This is just the semantics of the interpreter.

\begin{itemize}
    \item \textbf{skip}  returns the current environment
    \item \textbf{abort} returns \lstinline{#f}
    \item \textbf{begin} return the current environment
    \item \textbf{sequencing} folds over a list of statements and evaluates then to either an environment or aborts. 
    \item \textbf{assignment} assign a variables and return the current environment
    \item \textbf{conditional} either return the current environment after successful evaluation of condition and expression or return \lstinline{#f}
    \item \textbf{while} Either diverges or terminates with an environment or \lstinline{#f}
\end{itemize}
  
\end{questions}

\section{Hoare Logic for \imp (20 points)}\label{sec:imp:hoare}

The file \src[imp/]{hoare.rkt} contains our first verifier for \imp, and
\src[imp/]{proofs.rkt} shows how to use it to check proof outlines. This
verifier implements the Hoare logic rules presented in \lecture{11}, with some
small optimizations that make the verifier easier to use than a literal
implementation of the rules would be.

To enable checking of \imp programs with the Hoare verifier, we modify the \imp
grammar to include \lstinline{#:claim} annotations that specify Hoare logic
predicates to be checked:

{\tt\small
\begin{tabular}{lcl}
$S$ &:=&  $\ldots$ $|$ (\textbf{begin} $A^*$)  \\
$A$ &:=&  $S$ $|$ (\textbf{\#:claim} $C$)\\
\end{tabular}}

To pass the Hoare verifier, a program must include enough \lstinline{#:claim}
annotations to form a valid proof outline, as defined in \lecture{11}. In a
complete proof outline, each statement $S$ other than \lstinline{begin} is
surrounded by two claims, \lstinline{(#:claim $P$) $S$ (#:claim $Q$)}, such that
$\{P\}S\{Q\}$ is a valid Hoare triple. If a claim $P_1$ immediately precedes a
claim $P_2$, then it must be the case that $P_1 \implies P_2$. That is,
consecutive claims must be related by the Rule of Consequence. Finally, each
\lstinline{begin} statement should start and end with claims,
\lstinline{(begin (#:claim $P$) $S^*$ (#:claim $Q$))}, such that
$\{P\}S^*\{Q\}$ is a valid Hoare triple.


\begin{questions}
\item (18 points) \label{prob:hoare-proofs} Complete the proof outlines in
\src[imp/]{proofs.rkt} so that all included calls to the verifier succeed. Your
solution may add as many \lstinline{#:claim} annotations to a program as needed.
You may also wrap the bodies of procedures, \lstinline{while} loops, and
\lstinline{if} branches with \lstinline{begin} statements. But you may not
modify the code in any other way. Submit your copy of \src[imp/]{proofs.rkt}.



\item (2 points) \label{prob:hoare-study} The rules in \src[imp/]{hoare.rkt}
allow you to omit some annotations that would be required if the verifier used
the rules given in lecture. Explain briefly (in a few sentences) how the
implemented rules differ from the presented ones, and why these optimizations
are correct. \\

\textbf{Answer:} We exclude annotations for \lstinline{skip} and \lstinline{abort}. Since abort always return false it doesn't matter what the precondition. It takes any precondition and returns false. That is the postcondition always fails. \lstinline{skip} does nothing thus requires that the pre and postconditions are the same. This is a direct implication that is vacously satisfied. Thus it is only necessary to verify that consequence holds. 






\end{questions}

\section{WP, SP, and SE for \imp (30 points)}\label{sec:imp:wp/sp/se}

As you saw in the previous section, fully annotating programs is a lot of work!
Ideally, we would write just the annotations that require human
insight---preconditions, postconditions, and loop invariants---and offload the
rest of the work to an automated tool. So, in this section, you will develop key
parts of three such tools: a verifier based on weakest preconditions (WP), a
verifier based on strongest postconditions (SP), and a bounded verifier based on
symbolic execution (SE).

The code for all three tools can be found in \src[imp/]{tools.rkt}. The tools
take as input \imp programs that are annotated with preconditions,
postconditions, and (optionally) loop invariants according to the following
grammar, which extends the grammar from Section~\ref{sec:imp}:

{\tt\small
\begin{tabular}{lcl}
$P$ &:=&  (\textbf{procedure} $f$ ($x^*$) \textbf{:} ($x^*$) (\textbf{\#:requires} $C$)  (\textbf{\#:ensures} $C$) $S$) \\
$S$ &:=&  $\ldots$ $|$  (\textbf{while} $C$ (\textbf{\#:invariant} $C$) $S$) \\
\end{tabular}}

The file \src[imp/]{verified.rkt} contains sample programs in the extended \imp\
language.

Following  the approach from \lecture{12}, the tools do not work on this source
language directly. Instead, they first transform \imp programs into loop-free
code in an intermediate verification language, \ivl, which drops loops and
annotations from \imp and includes three new statements: \lstinline{(assert $C$)},
\lstinline{(assume $C$)}, and \lstinline{(havoc $x$)}. Then, they compute
verification conditions for the resulting program, and discharge them with
Z3.\looseness=-1


Problems \ref{prob:ivl-inv}-\ref{prob:ivl-unroll} ask you to implement two
different translations from \imp\ to \ivl, one for unbounded (WP/SP)
verification and one for bounded (SE) verification. The file \src[imp/]{ivl.rkt}
contains the skeleton code for both translations. When you complete the code,
all tool invocations in \src[imp/]{verified.rkt} should produce the expected
results. Submit your copy of \src[imp/]{ivl.rkt} as the answer to these
problems.

Problem \ref{prob:ver-study} asks you to study the code in \src[imp/]{tools.rkt}
and answer two questions about it. Include the answers to these questions in
your \texttt{hw3.pdf} submission.


\begin{questions}

\item (10 points) \label{prob:ivl-inv} The WP and SP verifiers expect the input
program to be annotated with a precondition, postcondition, and a loop invariant
for every loop in the program. Given such a program, they call the procedure
\lstinline{cut} from \src[imp/]{ivl.rkt} to obtain the corresponding \ivl\
program. Complete the implementation of \lstinline{cut} to soundly eliminate all
\lstinline{while} loops as shown in \lecture{12}.



\item (10 points) \label{prob:ivl-unroll} The SE bounded verifier expects the
input program to be annotated with a precondition and a postcondition, but it
requires no loop invariants.  Instead, it takes as input a non-negative integer
$k$ and unrolls every loop $k$ times to make the program finite. The unrolling
transformation is performed by the procedure \lstinline{unroll}. Complete the
implementation of \lstinline{unroll}  so that the resulting \ivl program can
perform up to $k$ iterations of every loop in the original \imp program,
ensuring that  there are no inputs on which the transformed program fails and
the original one does not. This means that \lstinline{unroll} cannot insert
unwinding assertions into the transformed program, as we did in \lecture{5}.



\item (10 points) \label{prob:ver-study} Study the code in \src[imp/]{tools.rkt}
and answer the following questions.

\begin{enumerate}
\item The procedure \lstinline{wp} calculates the weakest precondition of
 \lstinline{havoc} using a different rule than the one shown in \lecture{12}.
 Briefly explain how the rules differ, and why the implemented one is
 correct.\looseness=-1
 \\
 
 \textbf{Answer:} In \lecture{12} the \lstinline{havoc} weakest precondition is calculated by quantifying over all the value in the postcondition. However, in the homework, this is done by using a single symbolic constant that represent havocable. This doesn't require explicit quantification over all possible values in the postcondition.


\item The procedure \lstinline{interpretS+} extends the \imp\ interpreter from
 \src[imp/]{imp.rkt} to give meaning to \lstinline{assume}, \lstinline{assert},
 and \lstinline{havoc} statements. The SP verifier and the SE bounded verifer
 then use \rosette\ to turn this extended interpreter into a symbolic execution
 engine for \ivl. The implementation of \lstinline{assert} and \lstinline{havoc}
 is straightforward, as it relies on the corresponding constructs provided by
 \rosette. But the implementation of \lstinline{assume} is different.
 Briefly answer the following questions about this implementation:
 \begin{enumerate}
  \item How does \lstinline{interpretS+} implement \lstinline{assume}, and why
  is this a correct implementation of its semantics?\\
  
 \textbf{Answer:} \lstinline{interpretS+} uses \lstinline{and} which according to the Racket documentation will return \lstinline{#f} if the expression fails, otherwise it will have return the evaluation of the next expression which in this case is the environment . This is the semantics we want for \lstinline{assume}. If the expression fails we don't want the rest of the  test to be evaluated. \\

 
  \item How can we implement the \imp\ \lstinline{assume} using \rosette's
  \lstinline{assume}?  Your answer should take the form of a Rosette code
  snippet that should replace the given implementation.\\
  
 \textbf{Answer:} \lstinline{[ `(assume , C) (assume (interpret C env)) (assert #t) env]}

 
 \end{enumerate}

 \end{enumerate}




\end{questions}

\section{Verifying Programs with Dafny (40 points)}

In this part of the assignment, you will use \dafny (\lecture[pptx]{13}) to develop a
small library of verified procedures for operating on sequences. To get started,
install the \href{https://code.visualstudio.com}{Visual Studio Code IDE} and
then follow the instructions for installing the
\href{https://github.com/dafny-lang/dafny/wiki/INSTALL#visual-studio-code}{Dafny
extension} for this IDE. When you first open the skeleton solution file
\src[seq/]{seqlib.dfy} in the IDE, you will see warnings and errors issued by
the verifier. Once you have successfully completed the assignment, these
warnings and errors will go away. Submit your copy of \src[seq/]{seqlib.dfy} as
the solution to the problems in this section.

Before you start working on the problems, you may want to read  the
\href{https://dafny-lang.github.io/dafny/OnlineTutorial/guide}{Dafny Guide} and the
\href{https://dafny-lang.github.io/dafny/DafnyRef/DafnyRef#sec-sequences}{Sequences}
section of the \href{https://dafny-lang.github.io/dafny/DafnyRef/DafnyRef#sec-sequences}{Dafny Reference Manual}.
You may also want to read other parts of the manual.  You will need to know the following
constructs for this assignment: \lstinline{requires}, \lstinline{ensures},
\lstinline{modifies}, \lstinline{invariant}, and \lstinline{old}.

You can solve these problems in any order you like.  But we strongly recommend
that you do them in the order given in the assignment and the skeleton solution.
The problems ask you to write both proof annotations and code, and the comments
in the solution skeleton specify what may be changed and what must remain the
same.  Follow the directions carefully. You may implement the missing code
however you like: if it verifies, we'll take it!  But making liberal use of the
previously implemented components will lead to shortest solutions, with a few
lines of code and annotation as opposed to potentially many. You may also find
it easiest to comment out the irrelevant parts of the skeleton while working on
a given problem, in order to suppress spurious warnings and make verification
faster.

\begin{questions}

\item (10 points) \label{prob:dafny-first} The file  \src[seq/]{seqlib.dfy}
contains a partial implementation and specification of three functions for
operating on finite sequences of elements:

\begin{itemize}
\item \lstinline{sreverse($s$)} returns a sequence that reverses the order of
elements in the input sequence $s$: \lstinline{sreverse([1,2,3]) == [3,2,1]}.

\item \lstinline{ssubreverse($s, \mathit{start}, \mathit{end}$)} returns a
sequence that is like $s$ except for reversing the subsequence between the
$\mathit{start}$ index, inclusive, and $\mathit{end}$, exclusive:
\lstinline{ssubreverse([1,2,3],0,2) == [2,1,3]}.

\item \lstinline{srotate($s, k$)} returns a sequence that concatenates the last
$k$ elements of $s$ with the first $|s|-k$ elements of $s$:
\lstinline{srotate([1,2,3,4], 2) == [3,4,1,2]}.
\end{itemize}

Each function is equipped with a lemma and a client procedure that tests it on
specific inputs. Follow the instructions in  \src[seq/]{seqlib.dfy} to complete
the implementation and specification of these functions so that their lemmas and
clients are all verified.  The client code and the provided annotations may not
be changed, and more annotations may be added to a lemma only when stated in the
comments.



\item (20 points) Now that we have a clean functional implementation of our
sequence operations, we would also like to develop an efficient imperative
implementation that works on arrays.  The method \lstinline{subreverse}, for
example, uses a linear number of in-place \lstinline{swap} operations to
implement subsequence reversal. In fact, we specify the postcondition of
\lstinline{subreverse} by describing its effect on the state of the input array,
when viewed as a sequence of elements. Add enough annotations to
\lstinline{swap} and \lstinline{subreverse} so that Dafny can verify both of
them, as well as as the \lstinline{subreverse} client procedure.  Do not change
any of the provided code or annotations.



\item (10 points)  \label{prob:dafny-last} As a final step, complete the
array-based implementation and specification of array \lstinline{reverse} and
\lstinline{rotate}. Both implementations should run in linear time and constant
space.  Add enough annotations to \lstinline{reverse} and \lstinline{rotate} so
that Dafny can verify both of them, as well as their client procedures.  Do not
change any of the provided code or annotations.



\end{questions}

\end{document}
